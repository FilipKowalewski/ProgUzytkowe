\documentclass[a4paper,12pt]{article}
\usepackage[MeX]{polski}
\usepackage[utf8]{inputenc}

%opening
\title{Wydzial Matematyki i Informatyki uniwersytetu warminsko-mazurskiego}
\author{Filip Kowalewski}

\begin{document}

\maketitle
Historia Wydzialu

\begin{abstract}
Wydzia� Matematyki i Informatyki zosta� utworzony 1 wrze�nia 2001 roku, po powo�aniu dwa lata wcze�niej
Uniwersytetu Warmi�sko-Mazurskiego, ale jego korzenie si�gaj� lat 50. XX w. Decyzj� o powo�aniu Wydzia�u
podj�� Senat UWM w dniu 10 lipca 2001 r. Badania zwi�zane z zastosowaniami matematyki rozpocz�y si�
wraz z powo�aniem w 1950 roku Zak�adu Matematyki w Zespo�owej Katedrze Fizyki, a od 1951 roku �
Katedry Statystyki Matematycznej �wczesnej Wy�szej Szko�y Rolniczej, przemianowanej w 1972 r. na
Akademi� Rolniczo Techniczn�. Natomiast kszta�cenie matematyczne i badania w dziedzinie matematyki
zapocz�tkowane zosta�y wraz z utworzeniem w roku 1969 Wy�szej Szko�y Nauczycielskiej (od 1974 r. pod
nazw� Wy�sza Szko�a Pedagogiczna). Wydzia� jest kontynuatorem dzia�a� Katedry Zastosowa� Matematyki
ART oraz Instytutu Matematyki i Fizyki WSP.\ref{sec:tekst}


\end{abstract}
Od 27 kwietnia 2009 Wydzia�owi przyznano prawo do nadawania stopnia naukowego doktora w dziedzinie
nauk matematycznych w dyscyplinie matematyka[3]
.

\section{}

\end{document}