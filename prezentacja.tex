\documentclass[]{beamer}
%\usepackage[MeX]{polski}
%\usepackage[cp1250]{inputenc}
\usepackage{graphicx}
\usepackage{polski}
\usepackage[utf8]{inputenc}
\beamersetaveragebackground{blue!10}
\usetheme{Warsaw}
\usecolortheme[rgb={0.1,0.5,0.7}]{structure}
\usepackage{beamerthemesplit}
\usepackage{multirow}
\usepackage{multicol}
\usepackage{array}
\usepackage{graphicx}
\usepackage{enumerate}
\usepackage{amsmath} %pakiet matematyczny
\usepackage{amssymb} %pakiet dodatkowych symboli

\title{Jeziorany}
\date{}

\begin{document}

\frame
{
\maketitle
}

\begin{frame}
Jeziorany (dawniej Zybork, niem. Seeburg) - miasto w woj. warmińsko-mazurskim, w powiecie olsztyńskim, na Warmii nad rzeką Symsarną, siedziba gminy miejsko-wiejskiej Jeziorany. W latach 1975–1998 miasto administracyjnie należało do woj. olsztyńskiego.
\end{frame}

\begin{frame}
\frametitle{Spis treści}
\tableofcontents
\end{frame}


\section{Historia}
\begin{frame}
\begin{itemize}
\item W pobliżu Jezioran, koło wsi Krokowo, znajdowały się osady pruskie. Prawdopodobnie na wzgórzu, zwanym Świętą Górą (179 m n.p.m.), w czasach Prusów miały miejsce obrzędy religijne.
\pause
\item Założycielem Jezioran był biskup warmiński Herman z Pragi, który położył duże zasługi przy uregulowaniu granic biskupstwa i dominium warmińskiego oraz zasiedleniu południowej Warmii.
\end{itemize}
\end{frame}

\section{Demografia}
\begin{frame}
\begin{itemize}
\item Piramida wieku mieszkańców Jezioran w 2014 roku.
\end{itemize}
\begin{figure}
\centering
\includegraphics[width=0.7\hsize]{1.png}
\end{figure}
\end{frame}

\section{Zabytki}
\begin{frame}
\begin{itemize}
\item gotycki kościół św. Bartłomieja z 1390 r.
\pause
\begin{figure}
\centering
\includegraphics[width=0.2\hsize]{kosciol.jpg}
\end{figure}
\pause
\item resztki piwnic zamku biskupów warmińskich
\pause
\item kaplica św. Krzyża 
\pause 
\item rynek i pozostałości rzadko występującego rynku przelotowego
\end{itemize}
\end{frame}

\section{Transport}
\begin{frame}
Przez miasto przechodzą drogi:
\begin{itemize}
\item Droga wojewódzka nr 593 Miłakowo - Dobre Miasto - Jeziorany - Reszel
\pause
\item Droga wojewódzka nr 595 Jeziorany - Barczewo
\end{itemize}
\pause
Ok. 9 km na wschód od miasta funkcjonuje lądowisko Kikity.
\end{frame}

\section{Ciekawostki}
\begin{frame}
\begin{itemize}
\item W mieście nagrywano serial pt. "Stacyjka" (2004)
\pause
\item w 1562 r. w Jezioranach urodził się Jan Leo, autor książki "Dzieje Prus", ksiądz katolicki, kanonik w Dobrym Mieście.
\pause
\item W Jezioranach urodził się Aleksander Szczygło - były szef Biura Bezpieczeństwa Narodowego.
\pause
\item W szkole w Jezioranach uczył się niemiecki wynalazca Hermann Ganswindt. Urodził się w pobliskiej wsi Wójtówko.

\end{itemize}
\end{frame}



\end{document}