\documentclass[a4paper,12pt]{article}
\usepackage[MeX]{polski}
\usepackage[utf8]{inputenc}



\begin{document}
\title{Wzory matematyczne}
\author{Filiip Kowalewski}
\maketitle

Tutaj beda rózne wzory i zadania.

\tableofcontents 

\section{Wzór na delte}

$$
\Delta = b^2-4*a*c
$$

\section{Wzory skróconego mnozenia}

$$
a^2-b^2=(a-b)(a+b)\\
$$
$$
(a+b)^2=a^2+2ab+b^2\\
$$
$$
(a-b)^2=a^2-2ab+b^2\\
$$

\section{Zadania}

Oblicz:

$$ 5^8*\frac{1}{16^2} $$\
$$ \sqrt[3]{54} $$\
$$ (x*\sqrt{2}-2)^2=(2+\sqrt{2})^2 $$\
$$ a_n=\frac{(n^2-10)(2-3n)}{2n^3+n^2+3} $$\
$$ \frac{2^{k}}{2^{k+2}} $$\
$$ \frac{x^2}{2(x+2)(x-2)^3} $$\
$$ x=[x_1,x_2,... x_N] $$\
$$ \log_2{2^8} $$\
$$ \sqrt [3]{e^x-\log_2{x}} $$\

\section{Całki}

$$ \int_{-\infty}^{\infty} e^{-x^2}dx $$\
$$ \sum_{k=1}^{N} \frac{k*sin(k)}{2^{k}} $$\
$$ \sum_{i=1}^N \sum_{j=1}^N i*j $$\

\section{Inne wzory}

$$ \lim\limits_{n \to \infty} \sum_{k=1}^{n} \frac{1}{k^2} = \frac{\pi^2}{6} $$\
$$ \prod_{i=2}^{n=i^2} = \frac{\lim\limits^{n \to 4} (1+\frac{1}{n})^n}{\sum k(\frac{1}{n}} $$\
\end{document}